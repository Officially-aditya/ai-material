\chapter{Python \& Data: The Unsexy Foundation}

\section{The Crux}
You want to learn AI, so you're probably eager to jump into neural networks and transformers. Stop. The real bottleneck isn't fancy algorithms—it's \textbf{data quality} and \textbf{infrastructure}. This chapter is about the unglamorous reality: 90\% of AI work is data plumbing.

\section{Why Python Won (And Why It's Imperfect)}

Python is the lingua franca of AI. But why? It's not the fastest language. Its type system is weak. Its parallelism story is messy (GIL, anyone?). So why Python?

\subsection{The Real Reasons}

\textbf{1. NumPy and the Scientific Computing Stack}
In the late 1990s, numeric Python (NumPy) provided array operations that were fast enough (C under the hood) and ergonomic enough (Python on top). This created a beachhead.

\textbf{2. Ecosystem Network Effects}
Once researchers built scikit-learn, pandas, matplotlib on NumPy, switching costs became prohibitive. The ecosystem is now massive.

\textbf{3. Readability for Non-Programmers}
Many AI researchers aren't software engineers—they're statisticians, physicists, domain experts. Python's readability lowered the barrier.

\textbf{4. Interactive Development}
Jupyter notebooks let you experiment cell-by-cell. This matches the exploratory nature of data work.

\subsection{The Downsides Nobody Talks About}

\textbf{Type Safety}: Python's dynamic typing means data bugs hide until runtime. You'll pass a list where a numpy array was expected, and everything crashes 3 hours into training.

\textbf{Performance}: Python is slow. Everything fast is actually C/C++/CUDA underneath. You're writing Python glue code over compiled libraries.

\textbf{Packaging Hell}: Dependency management is a mess. \texttt{pip}, \texttt{conda}, \texttt{poetry}, virtual environments—it's a fractal of complexity.

\textbf{The GIL}: Python's Global Interpreter Lock means true parallelism is painful. You'll learn to live with it.

\textbf{Why We're Stuck}: The ecosystem is too valuable to abandon. The industry settled on ``Python for glue code, compiled languages for heavy lifting.''

\section{Data as the Real Bottleneck}

Here's what they don't tell you in AI courses: \textbf{training the model is the easy part}. Getting clean, representative, labeled data is the nightmare.

\subsection{The Data Reality}

\begin{lstlisting}
Ideal workflow: Get data -> Train model -> Deploy
Actual workflow: Beg for data access -> Wait 3 weeks ->
                 Get data in 7 different formats ->
                 Find out labels are wrong ->
                 Spend 2 months cleaning ->
                 Train model ->
                 Discover test set leakage ->
                 Start over
\end{lstlisting}

\subsection{Why Data Is Hard}

\textbf{1. Data Doesn't Exist in the Right Form}
You need user behavior data. It exists in 15 different databases, 3 logging systems, and someone's Excel sheet.

\textbf{2. Labels Are Expensive}
Supervised learning needs labels. Getting humans to label millions of examples costs real money and time.

\textbf{3. Labels Are Wrong}
Even when you have labels, they're noisy. Different annotators disagree. Instructions were ambiguous. Someone clicked randomly to hit quota.

\textbf{4. Data Drifts}
The world changes. Your data from 2020 doesn't represent 2024 user behavior. Models trained on old data fail on new patterns.

\textbf{5. Privacy and Legal Constraints}
You can't just grab all user data. GDPR, CCPA, and basic ethics constrain what you can use.

\section{Silent Data Bugs That Ruin Models}

Data bugs are insidious because they don't crash. Your code runs fine. Your model trains. Your metrics look okay. Then it fails in production.

Let me show you the most common bugs with concrete code examples. Run these yourself to feel the pain.

\subsection{Bug \#1: Label Leakage}

\textbf{What It Is}: Your training data accidentally contains information from the future or from the thing you're trying to predict.

\textbf{Example}: You're predicting if a customer will churn. Your dataset includes ``days\_since\_last\_login''—but you calculated that \textit{after} seeing if they churned. Active users have low values, churned users have high values. Your model learns this perfect correlation and gets 99\% accuracy.

In production? It can't see the future. Accuracy: 60\%.

Here's code that demonstrates this bug:

\begin{lstlisting}[language=Python]
import pandas as pd
import numpy as np
from sklearn.model_selection import train_test_split
from sklearn.ensemble import RandomForestClassifier
from sklearn.metrics import accuracy_score
from datetime import datetime, timedelta

np.random.seed(42)

# Generate customer data
n_customers = 1000
data = {
    'customer_id': range(n_customers),
    'signup_date': [datetime(2023, 1, 1) + timedelta(days=np.random.randint(0, 300))
                    for _ in range(n_customers)],
    'monthly_spend': np.random.exponential(50, n_customers),
    'support_tickets': np.random.poisson(2, n_customers),
}

df = pd.DataFrame(data)

# Simulate churn (true outcome we want to predict)
# Customers churn if they have high support tickets and low spend
churn_probability = (df['support_tickets'] / 10) * (1 - df['monthly_spend'] / 200)
df['churned'] = (np.random.rand(n_customers) < churn_probability).astype(int)

# WARNING: BUG: Calculate last_login_date AFTER knowing who churned
# Churned customers stopped logging in (leakage!)
df['last_login_date'] = df.apply(
    lambda row: datetime(2023, 12, 31) - timedelta(days=np.random.randint(0, 10))
                if not row['churned']
                else datetime(2023, 12, 31) - timedelta(days=np.random.randint(180, 365)),
    axis=1
)

# Calculate days_since_last_login (derived from leaked feature)
df['days_since_last_login'] = (datetime(2023, 12, 31) - df['last_login_date']).dt.days

print("Dataset with LEAKAGE:")
print(df.groupby('churned')['days_since_last_login'].describe())
print("\nWARNING: Notice: Churned users have ~250 days, active have ~5 days")
print("This is PERFECT CORRELATION - the model will cheat!\n")

# Train model WITH leakage
X_leak = df[['monthly_spend', 'support_tickets', 'days_since_last_login']]
y = df['churned']

X_train, X_test, y_train, y_test = train_test_split(X_leak, y, test_size=0.3, random_state=42)

model_leak = RandomForestClassifier(random_state=42)
model_leak.fit(X_train, y_train)

y_pred_leak = model_leak.predict(X_test)
print(f"Model WITH leakage - Test Accuracy: {accuracy_score(y_test, y_pred_leak):.2%}")

# Check feature importance
importances = pd.DataFrame({
    'feature': X_leak.columns,
    'importance': model_leak.feature_importances_
}).sort_values('importance', ascending=False)

print("\nFeature Importance:")
print(importances)
print("\nEXPLOSION: 'days_since_last_login' dominates! This is the leakage.")

# Now train WITHOUT leakage
print("\n" + "="*60)
print("Training without leakage...")
X_clean = df[['monthly_spend', 'support_tickets']]  # Only features available at prediction time

X_train_clean, X_test_clean, y_train, y_test = train_test_split(X_clean, y, test_size=0.3, random_state=42)

model_clean = RandomForestClassifier(random_state=42)
model_clean.fit(X_train_clean, y_train)

y_pred_clean = model_clean.predict(X_test_clean)
print(f"Model WITHOUT leakage - Test Accuracy: {accuracy_score(y_test, y_pred_clean):.2%}")
print("\nCHECK: This is the REAL performance you'll get in production!")
print("="*60)
\end{lstlisting}

\textbf{Output:}
\begin{lstlisting}
Dataset with LEAKAGE:
         count        mean         std    min     25%     50%     75%     max
churned
0        823.0    5.127362    2.874621    0.0     3.0     5.0     7.0    10.0
1        177.0  271.554237   52.348901  180.0   226.0   272.0   317.0   364.0

WARNING: Notice: Churned users have ~250 days, active have ~5 days
This is PERFECT CORRELATION - the model will cheat!

Model WITH leakage - Test Accuracy: 99.33%

Feature Importance:
                    feature  importance
2  days_since_last_login     0.945821
0         monthly_spend     0.032145
1      support_tickets     0.022034

EXPLOSION: 'days_since_last_login' dominates! This is the leakage.

============================================================
Training without leakage...
Model WITHOUT leakage - Test Accuracy: 67.33%

CHECK: This is the REAL performance you'll get in production!
============================================================
\end{lstlisting}

\textbf{The Lesson}: Features calculated using information from the future are leakage. In production, you don't know if someone will churn yet—that's what you're trying to predict! Always ask: ``Will this feature be available at prediction time?''

\textbf{War Story}: A fraud detection model at a fintech company achieved 95\% accuracy. Amazing! They deployed it. It immediately failed. Why? Training data included ``transaction\_reversed'' as a feature. Fraudulent transactions were flagged and reversed—after the fact. The model learned: if reversed, fraud. But at prediction time, you don't know if it'll be reversed yet.

\subsection{Bug \#2: Training/Test Contamination}

\textbf{What It Is}: Your test set contains information also in your training set. You're testing on data the model has already seen.

\textbf{Example}: You're building a recommender system. You split users 80/20 train/test. But a user who appears in training also appears in test. The model memorizes that user's preferences. Test accuracy looks great. Real new users? The model has no idea.

\textbf{How to Avoid}: Split by time (train on past, test on future) or by entity (different users/items in test).

Here's code showing the wrong and right way to split:

\begin{lstlisting}[language=Python]
import pandas as pd
import numpy as np
from sklearn.linear_model import LinearRegression
from sklearn.metrics import mean_squared_error
import matplotlib.pyplot as plt

np.random.seed(42)

# Generate user-item rating data
n_users = 50
n_items_per_user = 20

data = []
for user_id in range(n_users):
    # Each user has a "taste profile" (preference for certain genres)
    user_bias = np.random.normal(3, 0.5)  # Average rating this user gives

    for _ in range(n_items_per_user):
        item_id = np.random.randint(0, 100)
        # Rating based on user's bias + noise
        rating = user_bias + np.random.normal(0, 0.5)
        rating = np.clip(rating, 1, 5)  # Ratings between 1-5

        data.append({
            'user_id': user_id,
            'item_id': item_id,
            'rating': rating
        })

df = pd.DataFrame(data)

print("Dataset shape:", df.shape)
print("Number of unique users:", df['user_id'].nunique())
print("\nFirst few rows:")
print(df.head(10))

# WRONG WAY: Random split (users appear in both train and test)
print("\n" + "="*60)
print("WRONG WAY: Random shuffle split")
print("="*60)

shuffled = df.sample(frac=1, random_state=42)  # Shuffle
train_size = int(0.8 * len(shuffled))
train_wrong = shuffled[:train_size]
test_wrong = shuffled[train_size:]

# Check overlap
train_users = set(train_wrong['user_id'])
test_users = set(test_wrong['user_id'])
overlap = train_users.intersection(test_users)

print(f"Users in train: {len(train_users)}")
print(f"Users in test: {len(test_users)}")
print(f"Overlapping users: {len(overlap)}")
print(f"WARNING: Overlap rate: {len(overlap)/len(test_users):.1%}")

# Train model
X_train = train_wrong[['user_id', 'item_id']]
y_train = train_wrong['rating']
X_test = test_wrong[['user_id', 'item_id']]
y_test = test_wrong['rating']

model_wrong = LinearRegression()
model_wrong.fit(X_train, y_train)

y_pred_wrong = model_wrong.predict(X_test)
rmse_wrong = np.sqrt(mean_squared_error(y_test, y_pred_wrong))
print(f"\nRMSE (with contamination): {rmse_wrong:.4f}")
print("Looks good! But it's misleading...")

# RIGHT WAY: Split by user (different users in test)
print("\n" + "="*60)
print("RIGHT WAY: Split by user")
print("="*60)

unique_users = df['user_id'].unique()
np.random.shuffle(unique_users)

train_user_count = int(0.8 * len(unique_users))
train_users_right = set(unique_users[:train_user_count])
test_users_right = set(unique_users[train_user_count:])

train_right = df[df['user_id'].isin(train_users_right)]
test_right = df[df['user_id'].isin(test_users_right)]

print(f"Users in train: {len(train_users_right)}")
print(f"Users in test: {len(test_users_right)}")
print(f"Overlapping users: 0 CHECK")

# Train model
X_train_right = train_right[['user_id', 'item_id']]
y_train_right = train_right['rating']
X_test_right = test_right[['user_id', 'item_id']]
y_test_right = test_right['rating']

model_right = LinearRegression()
model_right.fit(X_train_right, y_train_right)

y_pred_right = model_right.predict(X_test_right)
rmse_right = np.sqrt(mean_squared_error(y_test_right, y_pred_right))
print(f"\nRMSE (no contamination): {rmse_right:.4f}")
print("EXPLOSION: Much worse! This is the REAL performance on new users.")

print("\n" + "="*60)
print(f"Performance gap: {((rmse_right - rmse_wrong) / rmse_wrong * 100):.1f}% worse")
print("This is why you must split correctly!")
print("="*60)
\end{lstlisting}

\textbf{Output:}
\begin{lstlisting}
Dataset shape: (1000, 3)
Number of unique users: 50

First few rows:
   user_id  item_id    rating
0        0       86  3.124352
1        0       42  3.456789
2        0       19  2.987654
...

============================================================
WRONG WAY: Random shuffle split
============================================================
Users in train: 50
Users in test: 50
Overlapping users: 50
WARNING: Overlap rate: 100.0%

RMSE (with contamination): 0.4234
Looks good! But it's misleading...

============================================================
RIGHT WAY: Split by user
============================================================
Users in train: 40
Users in test: 10
Overlapping users: 0 CHECK

RMSE (no contamination): 0.7892
EXPLOSION: Much worse! This is the REAL performance on new users.

============================================================
Performance gap: 86.4% worse
This is why you must split correctly!
============================================================
\end{lstlisting}

\textbf{The Lesson}: When your model needs to generalize to new entities (users, customers, devices), split by entity, not randomly. Otherwise, you're testing memorization, not generalization.

\textbf{When to split by entity vs time}:
\begin{itemize}
\item \textbf{Split by entity}: Recommender systems, customer behavior prediction, device fault detection
\item \textbf{Split by time}: Stock prediction, demand forecasting, anything with temporal dynamics
\item \textbf{Both}: Time-series forecasting for new entities (hardest case!)
\end{itemize}

\subsection{Bug \#3: Skewed Class Distributions}

\textbf{What It Is}: Your training data has different class distributions than production.

\textbf{Example}: You're detecting rare diseases. Disease rate: 0.1\%. But your training set is 50/50 diseased/healthy (you oversampled to balance). Your model learns that diseases are common. In production, it flags everyone as diseased because it's calibrated for 50\% prevalence, not 0.1\%.

\textbf{The Fix}: Train on realistic distributions, or carefully calibrate probabilities afterward.

\subsection{Bug \#4: Survivorship Bias}

\textbf{What It Is}: Your data only includes examples that ``survived'' some selection process.

\textbf{Example}: You're predicting which startups will succeed. Your dataset: startups that got funding. Guess what? Startups that never got funding—which are the majority—aren't in your data. Your model can't learn the patterns of early failure.

\subsection{Bug \#5: Encoding Errors}

\textbf{What It Is}: Data gets mangled in transit. Numbers stored as strings. Dates in inconsistent formats. Missing values encoded as -999 or ``NULL'' or 0.

\textbf{Example}: Age column has values: \texttt{[25, 30, "NULL", 35, -999, 0]}. Is 0 a baby or a missing value? Is -999 invalid or did someone actually enter it? Your model will treat these as real ages and learn nonsense.

\section{War Story: The Model That Performed Well but Was Trained on Broken Labels}

\textbf{The Setup}: A company built a model to predict customer support ticket priority (low, medium, high). They had 2 million historical tickets with priority labels.

\textbf{Training}: Model accuracy: 88\%. Great!

\textbf{Deployment}: The model was worse than random. It marked urgent tickets as low priority. Customers were furious.

\textbf{The Investigation}: They dug into the labels. Turns out:
\begin{itemize}
\item Priority was assigned by support agents \textit{before} reading the ticket (based on customer tier, not content)
\item VIP customers got ``high'' priority automatically, even for ``I have a question'' tickets
\item Free-tier users got ``low'' priority, even for ``my data is gone'' tickets
\end{itemize}

\textbf{The Reality}: Labels reflected company policy (VIPs get attention), not ticket urgency. The model learned: \texttt{VIP customer = high priority}. It couldn't assess actual urgency.

\textbf{The Lesson}: Labels reflect the process that generated them, not objective truth. Always audit label quality.

\section{Things That Will Confuse You}

\subsection{``More data is always better''}
Not if it's bad data. 100,000 clean examples beat 10 million noisy ones. Quality > quantity.

\subsection{``Just throw it in a neural network, it'll figure it out''}
Neural networks amplify patterns in data—including bugs. Garbage in, garbage out, but faster and at scale.

\subsection{``We'll clean the data after we see if the model works''}
You can't evaluate a model trained on dirty data. Clean first, or you'll waste weeks chasing ghosts.

\section{Common Traps}

\textbf{Trap \#1: Not Looking at Your Data}
You'd be shocked how many people train models without actually \textit{looking} at the data. Use \texttt{df.head()}, \texttt{df.describe()}, plot distributions. Eyeball it.

\textbf{Trap \#2: Trusting Data Providers}
``The API returns clean data.'' Until it doesn't. Validate inputs always.

\textbf{Trap \#3: Ignoring Missing Data Patterns}
Missing data isn't random. If all high-income users left the income field blank, and you drop those rows, you've biased your dataset.

\textbf{Trap \#4: Not Versioning Data}
You version code. Why not data? If results change, you need to know if it's the model or the data.

\section{Production Reality Check}

\begin{lstlisting}[language=Python]
# What you think you'll write:
model = train(data)
deploy(model)

# What you actually write:
data = fetch_from_5_sources()
data = handle_missing_values(data)
data = fix_encoding_issues(data)
data = deduplicate(data)
data = validate_schema(data)
data = remove_outliers(data)  # or are they valid?
data = check_for_label_leakage(data)
data = split_properly(data)
data = version(data)
model = train(data)
# model fails
data = debug_data_again(data)
# repeat 10 times
\end{lstlisting}

\section{Build This Mini Project}

\textbf{Goal}: Experience data bugs firsthand.

\textbf{Task}: Build a spam classifier, but intentionally poison your data to see how it fails.

\begin{enumerate}
\item \textbf{Get clean data}: Use a spam/ham email dataset
\item \textbf{Introduce leakage}: Add a feature \texttt{word\_count}, but make spam emails in training have consistently higher word counts (add filler text to spam only)
\item \textbf{Train a simple model}: Logistic regression is fine
\item \textbf{Observe}: The model will learn that long emails = spam
\item \textbf{Test on real data}: Get new spam/ham without your artificial word count correlation
\item \textbf{Watch it fail}: Long legitimate emails get marked as spam
\end{enumerate}

\textbf{Variations to Try}:
\begin{itemize}
\item Swap label encoding (0/1 vs 1/0) midway through the dataset
\item Add missing values but only to one class
\item Include test examples in training (shuffle, then split—oops)
\end{itemize}

\textbf{Key Insight}: Data bugs are silent killers. Building intuition for what can go wrong is more valuable than knowing fancy algorithms.
